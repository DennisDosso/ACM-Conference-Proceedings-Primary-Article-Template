%%
%% This is file `sample-sigconf.tex',
%% generated with the docstrip utility.
%%
%% The original source files were:
%%
%% samples.dtx  (with options: `sigconf')
%% 
%% IMPORTANT NOTICE:
%% 
%% For the copyright see the source file.
%% 
%% Any modified versions of this file must be renamed
%% with new filenames distinct from sample-sigconf.tex.
%% 
%% For distribution of the original source see the terms
%% for copying and modification in the file samples.dtx.
%% 
%% This generated file may be distributed as long as the
%% original source files, as listed above, are part of the
%% same distribution. (The sources need not necessarily be
%% in the same archive or directory.)
%%
%% The first command in your LaTeX source must be the \documentclass command.
\documentclass[sigconf, nonacm]{acmart}
%\documentclass{sig-alternate-10pt}

%\paperwidth=8.5in
%\paperheight=11in
%\usepackage[margin=1in]{geometry}


\AtBeginDocument{%
  \providecommand\BibTeX{{%
    \normalfont B\kern-0.5em{\scshape i\kern-0.25em b}\kern-0.8em\TeX}}}



%%
%% Submission ID.
%% Use this when submitting an article to a sponsored event. You'll
%% receive a unique submission ID from the organizers
%% of the event, and this ID should be used as the parameter to this command.
%%\acmSubmissionID{123-A56-BU3}

%%
%% The majority of ACM publications use numbered citations and
%% references.  The command \citestyle{authoryear} switches to the
%% "author year" style.
%%
%% If you are preparing content for an event
%% sponsored by ACM SIGGRAPH, you must use the "author year" style of
%% citations and references.
%% Uncommenting
%% the next command will enable that style.
%%\citestyle{acmauthoryear}

\usepackage{url}

%%
%% end of the preamble, start of the body of the document source.
\begin{document}

%%
%% The "title" command has an optional parameter,
%% allowing the author to define a "short title" to be used in page headers.
\title{Information and Research Science connecting to Digital and Library Science}
\subtitle{Report on the 17th Italian Research Conference on Digital Libraries, IRCDL2021}

%\alignauthor 
%Lawrence P. Leipuner\\
%       \affaddr{Brookhaven Laboratories}\\
%       \affaddr{Brookhaven National Lab}\\
%       \affaddr{P.O. Box 5000}\\
%       \email{lleipuner@researchlabs.org}
       
  
\author{Dennis Dosso}
\email{dennis.dosso@unipd.it}

\affiliation{%
  \institution{University of Padova}
  \city{Padua}
  \country{Italy}
}

\author{Stefano Ferilli}
\email{stefano.ferilli@uniba.it}
\affiliation{%
  \institution{University of Bari}
  \city{Bari}
  \country{Italy}
}

\author{Paolo Manghi}
\email{manghi@isti.cnr.it}
\affiliation{%
  \institution{ISTI - Consiglio Nazionale delle Ricerche}
  \city{Pisa}
  \country{Italy}
}

\author{Antonella Poggi}
\email{antonella.poggi@uniroma1.it}
\affiliation{%
  \institution{``La Sapienza'', University of Rome}
  \city{Rome}
  \country{Italy}
}

\author{Giuseppe Serra}
\email{giuseppe.serra@uniud.it}
\affiliation{%
  \institution{University of Udine}
  \city{Udine}
  \country{Italy}
}

\author{Gianmaria Silvello}
\email{gianmaria.silvello@unipd.it}
\affiliation{%
  \institution{University of Padua}
  \city{Padova}
  \country{Italy}
}





%%
%% By default, the full list of authors will be used in the page
%% headers. Often, this list is too long, and will overlap
%% other information printed in the page headers. This command allows
%% the author to define a more concise list
%% of authors' names for this purpose.
%\renewcommand{\shortauthors}{Dosso, Ferilli et al.}

%%
%% The abstract is a short summary of the work to be presented in the
%% article.
%\begin{abstract}
%  A clear and well-documented \LaTeX\ document is presented as an
%  article formatted for publication by ACM in a conference proceedings
%  or journal publication. Based on the ``acmart'' document class, this
%  article presents and explains many of the common variations, as well
%  as many of the formatting elements an author may use in the
%  preparation of the documentation of their work.
%\end{abstract}


%%
%% Keywords. The author(s) should pick words that accurately describe
%% the work being presented. Separate the keywords with commas.
%\keywords{datasets, neural networks, gaze detection, text tagging}


%%
%% This command processes the author and affiliation and title
%% information and builds the first part of the formatted document.
\maketitle

\section{Introduction}
...

\section{Conference Contributions}
\subsubsection*{Invited talk: The swings and roundabouts of a decade of fun and games with Research Objects} Prof. Carole Goble discussed her work on the concept of Research Objects\footnote{\url{http://researchobject.org/}}\cite{bechhofer2013linked} in her keynote\footnote{\url{https://www.slideshare.net/carolegoble/the-swings-and-roundabouts-of-a-decade-of-fun-and-games-with-research-objects-242974091}}. We are today transitioning toward the fourth paradigm of science, where the research outcomes are more than just publications and data, but also include software, models, workflows, SOPs, lab protocols, etc. All these products of science should be considered as first class citizens of scholarship, and be treated following the FAIR and Reproducible principles (FAIR+R). 
The Research Object is a general framework, where research outcomes are related and bundled together. Their metadata describe their content and context, enriching them with additional information required to make them FAIR+R. These resources thus become shareable, citable, exchangeable, present versioning and snapshots, can be identified with tools such as PIDs and can be registered and deposited on their own in services like Zenodo, thus to be later unpackaged, accessed, activated, and reproduced if appropriate. 

\subsubsection*{Data and Platforms}
This section presented new platform to manage and share data, and shed light on important challenges faced by already existing and famous data infrastructures.
\citet{biasini2021fullbrain} presented FullBrain\footnote{\url{https://fullbrain.org/}}, a social e-learning platform where students can share and track their knowledge, helping each other in their learning process.
\citet{GargiuloGTZ21} note that the FAIR RDM initiatives in Italy are still based on communities of practice, thus investigate the perception of some leader initiative with respect to good practices, challenges and the strategic vision. 
\citet{BaglioniMMABB21} highlight the ``service misuses'' suffered by the ORCID infrastructure, that put in jeopardy its very mission.

\subsubsection*{Data Access and Monitoring}
This section presented new systems and techniques based on data and how its correct use can have meaningful impacts on the everyday life of people in different aspects of their life, ranging fro, learning to security.
\citet{ZanichelliETTC21} note that digital games can be a valuable tool for enhancing learning, and aims to understand how to improve the assessment of learning obtained through them.
\citet{AvanziCCG021} presented \emph{NoBis}, a new service for monitoring the crowding of indoor spaces during the COVID-19 era.
\citet{SpadiSD21} presented the first results of the project RESTORE, whose main goal is the recovery, integration, and accessibility of data and digital objects produced in the last twenty years by the partners of the project, thus to build a knowledge base on the history of the town of Prato.


\subsubsection*{Information and Knowledge}
This section presents new applications and measures to obtain and evaluate information. 
\citet{IrreraS21} addresses the problem of finding context information for news articles by extracting entities and relationships from documents that can provide context to documents, and use these entities to improve the effectiveness of ranking methods.
\citet{LancioniMP0T21} use Generative Adversarial Networks to predict keyphrases, i.e. short text sequences that convey the main semantic meaning of a document. 
\citet{Ferrante0P21} extend the AWARE measure with a set of supervised methods by using several TREC collections. 

\subsubsection*{Character Recognition}
This section presented new approaches to address problem connected to the automatic recognition of handwritten text to uncover texts still unreadable and provide new and meaningful tools to the experts.
\citet{FanteN21} propose a mixed qualitative-quantitative approach to OCR error detection and correction to develop a methodology for compiling historical corpora. 
\citet{HeilNH21} introduces Handwritten Text Recognition and Document Image Analysis approaches to address the challenges inherent to the original drafts of Astring Lindren, a Swedish author whose original drafts and manuscripts are edited in the Melin system of shorthands, as of today considered undecipherable. 
\citet{MarinaiPRS21} describe a system for the location of simple graphemes in mediaeval manuscripts based on the Mask R-CNN convolutional neural network, a first approach to provide paleographers with a tool to speed up and refine the process of dating and determining the origin of manuscripts.

\subsubsection*{Text Analysis}
In this section it is possible to find works that explore new potential applications connected to the analysis of documents, opening up new applications and research lines. 
The work proposed by \citet{BernasconiCM21} explores the idea of navigating the semantic relationships among extracted entities as a way to search a text corpus, and the evaluation carried with potential users has shown the feasibility and effectiveness of the approach.
\citet{Ferilli21} presents a paper which deals with architectural floorplans, proposing an approach based on formal representation and reasoning for their understanding and interpretation, opening up a viable and promising new line of research.  
\citet{DossoS21} proposed a new problem, called Data Credit Distribution, to annotate data in a database with a value, called credit, representing the impact and importance of this data in the scientific domain. 


\subsubsection*{DL Services}
\citet{BernasconiBCCGL21} work in the context of the NOTAE project, which investigates graphic symbols, visual units in written text, dating back to the Roman State and Post-Roman Kingdoms. The authors argue that a different, knowledge graph-based, approach to the task can ease the work of researchers at finding non trivial implications in the data. 
\citet{Svarre21} studies labels used by Danish academic library websites, which can support user interactions with library websites and their related content, shedding more light on the characteristics and variety of labels used across libraries. 
\citet{AlmeidaFM21} are developing the ROSSIO Thesaurus, whose objective is to support content discovery and management in the ROSSIO Portuguese research infrastructure for arts and humanities, and describe its modeling process and integration in the infrasctcture, together with its publication as Linked Open Data.


\section{Panel}
...



\begin{acks}
...
\end{acks}

%%
%% The next two lines define the bibliography style to be used, and
%% the bibliography file.
\bibliographystyle{ACM-Reference-Format}
\bibliography{ircdl_bib}

\end{document}
\endinput
%%
%% End of file `sample-sigconf.tex'.
